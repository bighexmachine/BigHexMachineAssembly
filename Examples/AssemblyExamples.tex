\documentclass[10pt,a4paper]{article}
\usepackage[utf8]{inputenc}
\usepackage{amsmath}
\usepackage{amsfonts}
\usepackage{amssymb}
\usepackage[left=2cm,right=2cm,top=1cm,bottom=2cm]{geometry}
\usepackage{listings}
\setlength\parindent{0pt}
\author{Samuel Russell}
\title{The BigHex Machine Assembly Examples}
\begin{document}
\maketitle


\section{Adding Two Constants}

First you must load the two constants in to the two registers (areg and breg) and then perform the interegister add operation. 
\\
\begin{lstlisting}[frame=single]
   LDAC  1
   LDBC  2
   OPR    ADD
\end{lstlisting}

The first line loads the number one in to areg, the second line loads two in to breg and the third adds together the value of areg and breg and puts the answer in to areg.
\\\\
Note that each instruction goes on a new line and has at least one space at the beginning.

\section{Adding Two Values from Memory}

A data portion is allocated in memory and the address of it (using a label) is used to load the value in to the registers before the add operation (used as before) and store back to memory afterwards.
\\
\begin{lstlisting}[frame=single]
   LDAM  Lval_one
   LDBM  Lval_two
   OPR   ADD
   STAM  Lval_out

Lval_one
   DATA   0x10
Lval_two
   DATA   2
Lval_out
   DATA   0
\end{lstlisting}

Note we can also use hexadecimal representation to define the values.
\newpage
\section{Summing the Integers from 1 to N}

For this operation we will create a loop using the conditional and unconditional branch instructions. We will need to keep track of the running total and a variable to count the number of iterations. Since we only have two registers we must store these values in our main memory and use loads and stores as we manipulated them.
We use labels to mark lines of code to jump to.
\\
\begin{lstlisting}[frame=single]
Lstart
 LDAM Lsum
 LDBM Lcounter
 OPR ADD
 STAM Lsum
 LDAM Lcounter
 LDBC 1
 OPR SUB
 STAM Lcounter
 BRZ Lend
 BR Lstart

Lend
 LDAM Lsum

Lcounter
 DATA 10
Lsum
 DATA 0
\end{lstlisting}

This code can be made shorter by utilising the fact that the loop counter is already in breg when we need to decrement it.
\\
\begin{lstlisting}[frame=single]
Lstart
 LDAM Lsum
 LDBM Lcounter
 OPR ADD
 STAM Lsum
 LDAC -1
 OPR ADD
 STAM Lcounter
 BRZ Lend
 BR Lstart

Lend
 LDAM Lsum

Lcounter
 DATA 10
Lsum
 DATA 0
\end{lstlisting}

\newpage
\section{Multiplication}

This machine has a very RISC architecture which means instructions can be decoded simply and the control logic is relatively small. This does mean however that there is no multiplication instruction and so it has to be done iteratively (this was true up until surprisingly recently).
\\
\begin{lstlisting}[frame=single]
Lstart
 LDAM Lsum
 LDBM Lval_a
 OPR ADD
 STAM Lsum
 LDAM Lval_b
 LDBC 1
 OPR SUB
 STAM Lval_b
 BRZ Lend
 BR Lstart

Lend
 LDAM Lsum

Lval_a
 DATA 0x2
Lval_b
 DATA 0x10
Lsum
 DATA 0
\end{lstlisting}

\newpage
\section{Factorial}

We can then use this multiplication code we have written to write a program that calculates the factorial of a given number. (remember $n! = (n)*(n-1)...(2)*(1) $ )

\begin{lstlisting}[frame=single]
-jump straight to code
 BR Lstart
-Allocate space to store variables
Lval_a
 DATA 0x0
Lval_b
 DATA 0x0
Lmult_sum
 DATA 0x0
Lcount
 DATA 4
Ltotal
 DATA 1

-this bit loads the parameters for our multiplier from our total variables
Lstart
 LDAM Ltotal
 STAM Lval_a
 LDAM Lcount
 STAM Lval_b
 LDAC 0
 STAM Lmult_sum
- this is the call to multiplier
 BR Lmult_start
-this bit receives the code from multiplier,
-then checks if we need to go round again.
Lcheck_finish
 LDAM Lmult_sum
 STAM Ltotal
 LDAM Lcount
 LDBC 1
 OPR SUB
 STAM Lcount
 BRZ Lend
 BR Lstart

-holds the processor in an infinite loop
Lend
 LDAM Ltotal
 BR Lend

-this is the multiplier code as before
Lmult_start
 LDAM Lmult_sum
 LDBM Lval_a
 OPR ADD
 STAM Lmult_sum
 LDAM Lval_b
 LDBC 1
 OPR SUB
 STAM Lval_b
 BRZ Lmult_end
 BR Lmult_start
-now we return to the previous code
Lmult_end
 BR Lcheck_finish
\end{lstlisting}
This piece of code calculates factorial 4. It works well but some improvements could be made to make it run faster.
\end{document}
